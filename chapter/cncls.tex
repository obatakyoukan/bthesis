\chapter{おわりに}
\label{ch:cncls}
\section{まとめ}
%アルゴリズムの空間計算量をまとめものが,表\ref{tab:result-mem}である.
隣接行列での空間計算量は,MCSも提案アルゴリズムも$O(n^2)$である.
%従来のMCSアルゴリズムでは,探索中の空間計算量$O(nw)$なので,グラフを隣接リストで持ったとしても
%クリークとグラフのサイズが近いようなときには,空間計算量が$O(n^2)$になる可能性があるが,
隣接リストでは,MCSでの空間計算量は$O(n \omega + m )$であるが,
提案アルゴリズムの空間計算量が$O(n+ m)$で抑えられることを示した.
これは,最大クリークのサイズが大きな場合でも,最大メモリの使用量が大きくならずに抑えれていることを表す.

%実験において,隣接リスト表現でも隣接行列表現と同じ分枝数であることを確認した.
%ほとんどのグラフに対して,実行時間は大きくなり,消費メモリは小さくなることを確認した.
提案アルゴリズムでは,ほとんどのグラフに対して実行時間が増加した.
解が大きな場合について,最大メモリ使用量が小さくなった.
近似解を利用した場合で近似解と解が一致した場合は,
探索木の深さが,クリークのサイズより小さくなる可能性があり.
最大メモリ使用量が小さくなることがある.
また,MCSではメモリ使用量が大きくなり動かないが,提案アルゴリズムでは動くようなグラフがあることも確認できた.

%メモリ使用量を抑える重要性についても確認することができた.

\begin{comment}
\begin{table}[htb]
\centering
\caption{アルゴリズムの空間計算量}
\label{tab:result-mem}
    \begin{tabular}{|l|l|l|}
\hline
 & MCSアルゴリズム &  提案アルゴリズム   \\ \hline
隣接行列 & $O(n^2)$ & $O(n^2)$  \\ \hline
隣接リスト & $O(n \omega +m )$ & $ O(n + m ) $   \\ \hline

\end{tabular}
\small{  $n$はグラフのサイズ$|V|$,$m$はグラフの辺の数$|E|$,$w$はグラフの最大クリークのサイズ }
\end{table}
\end{comment}

\section{今後の課題}
%今後の課題としては,以下の点が挙げられる.
%\begin{itemize}
%   \item 
提案アルゴリズムは,クリークサイズが小さいようなときはメモリ使用量の恩恵が少なく,実行時間が遅いだけである.
そのため,探索の木に近い部分では提案手法を,根にある程度近づくとMCSの手法に切り替えることで,
メモリ使用量を抑えつつ十分に効率的な時間で終了できるな工夫を考えることが今後の課題として考えられる.
%メモリ使用量を抑えつつ十分に効率の良い時間で終了できる工夫を盛り込みたいと考えました.
%	\item 
また,MCTなどの他のより効率的なアルゴリズムに対しての提案手法を考え方を組み込むことによるメモリ使用量を改善することも今後の課題として考えられる.
%\end{itemize}

