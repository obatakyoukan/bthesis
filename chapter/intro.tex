\chapter{はじめに}
\label{ch:intro}
\section{背景}
最大クリークを発見する問題は,NP困難問題のクラスに属する組合せ最適化問題の一つとして古くから研究されている問題である\cite{Karp1972}.
現実の様々な問題が,最大クリークの発見やそれに類する問題としてモデル化できることから,
最大クリークを発見する問題は,工学的に重要な問題となっている.

%応用例としては,経済学\cite{BOGINSKI20063171}や符号理論\cite{etzion1998greedy}などが挙げられる.
%最大クリークを抽出する高速なアルゴリズムは今なお研究されている.
たとえば,株式市場におけるマーケットグラフ(market graph)の最大クリークは重要な特徴を表す.
%マーケットグラフでの最大クリークは,株式市場での重要な特徴を表す.
%マーケットグラフは次のように構成される.
%頂点を株とし,2つの頂点が辺で接続されていることと2つの株価の相関係数が閾値を超えていることが一致している.
マーケットグラフとは,各株の銘柄を頂点とし,
株価の相関係数がある閾値を超える頂点どうしを辺で接続したグラフである.
閾値を高い値にすると,辺が存在することは,2つの株に重要な相関関係があることを表す.
%このグラフのクリークのどの頂点も同じクリークに属する他の頂点に似た振る舞いをしていることを表す.
グラフのクリークに含まれる頂点に対応する株は,似た振る舞いをしていることを表す.
したがって,最大クリークは,株式市場における似た振る舞いをする株の最大のグループという特徴を表す\cite{BOGINSKI20063171}.
\section{関連研究}
\label{sec:relation}
%無向グラフから最大クリークを1つ抽出する問題に関して実働上で効率良く解を得ることを目的としたアルゴリズムが日々研究され開発されている.

グラフから最大クリークの厳密解を求めるアプローチのほとんどは,分枝限定法を基にしている.
%異なることは,
厳密解を求める各アルゴリズムの違いは,
主に上界と下界を決定する手法と分岐の戦略である.
1990年に,CarraghanとPardalos \cite{carraghan1990exact}は,
%多くの後の分枝限定法の基礎となるシンプルな
2つの工夫を用いた単純なアルゴリズムを提案している.
1つ目は,常に次数最小の頂点を選べるように頂点を整列させることである.
2つ目は,クリークに全ての頂点を加えても最大クリークを更新できなくなったら探索を打ち切ることである.
%頂点を常に次数最小の順に選べるように頂点を整列させてから,クリークに全ての頂点を加えても最大クリークを更新できなくなったら探索を打ち切るという単純な分枝条件の
%アルゴリズムを提案している.
これは,後の厳密解を求める多くのアルゴリズムに影響を与えた.
同年,BabelとTinhofer \cite{Babel1990}は,彩色を用いることで最大クリークの上界の決定することによって分枝限定をより効率的に探索するアルゴリズムを提案している.
その後,2003年にTomita \cite{tomita2003efficient}らは,彩色による分枝限定法によるアルゴリズムでMCQが提案している.
さらに,Tomitaらはそれを改善したアルゴリズムとして,2007年にMCR \cite{tomita2007efficient}を,
2010年にMCS \cite{tomita2010simple}\cite{tomita2013simple}を,2016年にMCT \cite{tomita2016much}を提案している.
%また,それを改善したアルゴリズムとしてMCR(2007)\cite{tomita2007efficient},MCS(2010)\cite{tomita2010simple}\cite{tomita2013simple},MCT(2016)\cite{tomita2016much}などの高速な最大クリーク抽出アルゴリズムが提案されている.

一方,グラフから最大クリークの近似解を高速に求める手法が提案されている.
この手法のアプローチには,貪欲な手法%\cite{DBLP:journals/jgo/PardalosX94a}や
と局所探索法がある.
貪欲な手法の1つとして,2004年にGrossoらは,
DAGS \cite{DBLP:journals/heuristics/GrossoLC04}を提案している.
これは,現在のクリークをより有望なクリークに変換することと頂点に重みをつけるという2点の工夫を用いたアルゴリズムである.
局所探索法の1つとして,2005年にKatayamaらは,
$k$-opt Local Search(KLS) \cite{katayama2005effective}を提案している.
これは,クリークに頂点を追加・削除することで生成可能な解の集合を
近傍と捉えて探索を行うアルゴリズムである.

\section{目的}
%\section{内容}
%Tomitaらの提案しているMCQと,MCR,MCS,MCTは省メモリ化についてあまり研究をされていない.
MCSは,深さ優先探索に基づく最大クリークを発見するアルゴリズムである.
MCSのメモリ使用量のボトルネックは,深さ優先探索の各段階は,各段階の探索が終了まで,頂点集合などの情報を
コピーして保持する必要があることである.
このため,最大クリークが大きいとき,探索が深くなりやすいので,メモリ使用量も大きくなる.

本論文では,最大クリークが巨大なグラフに対しても実用上で解を得られるように
%本論文では巨大なグラフに対してもアルゴリズムが適用できるように,
%アルゴリズムの
最大メモリ使用量の削減に着目をして取り組んだ.%目的
%本論文では,
我々は,TomitaらのMCSに基づく,その省メモリ化の手法を提案する.
省メモリ化のアイデアとしては,深さ優先探索で子に進む際に捨てる情報のみを保持し,
子には,頂点集合の情報をコピーせずに渡すことで省メモリ化を行う.
%バックトラックした後に,彩色アルゴリズムを再び適用することで,探索順序を決定する.
MCTでは,部分問題に応じて彩色アルゴリズムを切り替える.
彩色アルゴリズムの1つは親の彩色を引き継ぐため,その情報を保存する必要がある.
このため,省メモリ化のアイデアを適用できないので,
MCSを改善したMCTではなく,MCSを基にした.

提案アルゴリズムと基にしたMCSに比較して,空間計算量や時間計算量が,改善したかどうかを調べる.
私がみる限り,Tomitaらは,MCSアルゴリズムに対して,空間計算量
%や時間計算量
を記載していない.
したがって,MCSと提案アルゴリズムの両方に対して,空間計算量と時間計算量についての解析を行う.
さらに,提案手法を実装し,35個のグラフに対して実験を行う.
実験に使用したグラフは,DIMACSベンチマークセットのグラフ\footnote{\url{http://iridia.ulb.ac.be/~fmascia/maximum_clique/DIMACS-benchmark}}
とクリークサイズの大きなグラフから辺をいくつか削除することで作成したグラフ, 
DIMACSのグラフのいくつかのグラフを合体させて,そのグラフ間に辺をランダムに引くことによって作成したグラフ
の3種類を用意した.

\section{結果}
 $n$はグラフのサイズ,$m$は辺の本数,$\omega$を最大クリークのサイズとする.

本論文の主な貢献は,以下の5点である.
\begin{itemize}
    \item MCSアルゴリズムの隣接行列表現から隣接リスト表現へ適用した.
    \item MCSアルゴリズムを基にした省メモリなアルゴリズムを提案をした.
    \item MCSアルゴリズムと提案アルゴリズムの空間計算量について解析を行った.
	空間計算量について,まとめたものが,表\ref{tab:result-mem}である.
	提案アルゴリズムの隣接リスト表現の空間計算量$O(n+m)$が,MCSアルゴリズムの$O(n \omega + m )$に対して小さく抑えられることを示した.
    %\item MCSアルゴリズムと提案アルゴリズムの再帰処理の時間計算量についての解析を行った.
%	提案アルゴリズムの再帰処理の時間計算量が隣接行列で$O(n^4)$時間である.
%	これは,MCSアルゴリズムの$O(n^3)$時間に対して悪化している.
%	また,隣接リスト表現についても提案アルゴリズムは$O(n^4 \log n )$時間である.
%	これは,MCSアルゴリズムの$O(n^3 \log n )$時間に対して悪化している.
    \item %MCSアルゴリズムの再帰処理の時間計算量は,隣接行列表現で$O(n^3)$時間,隣接リスト表現で$O(n^3 \log n )$時間であることを示した.
	%さらに,
	提案アルゴリズムの再帰処理の時間計算量は,隣接行列表現で$O(n^4)$時間,隣接リスト表現で$O(n^4 \log n)$時間であることを示した.

    \item MCSアルゴリズムと提案アルゴリズムの2つのアルゴリズムに対して,35個のグラフでの実験を行った.
	最大クリークのサイズの大きい14個のグラフに対して,最大メモリ使用量の削減を確認し,その効果に対しての考察を行なった.
	%
実験に使用したグラフは,DIMACSベンチマークセットのグラフ\footnotemark[1]
とクリークサイズの大きなグラフから辺をいくつか削除することで作成したグラフ, 
DIMACSのグラフのいくつかのグラフを合体させて,そのグラフ間に辺をランダムに引くことによって作成したグラフ
の3種類を用意した.

\end{itemize}
 %$n,m,w$では,
%関連研究として,鳥谷部\cite{toriyabe}は○○を発見した.

\begin{table}[p]
\centering
\caption{アルゴリズムの空間計算量}
\label{tab:result-mem}
    \begin{tabular}{|l|l|l|}
\hline
 & MCSアルゴリズム &  提案アルゴリズム   \\ \hline
隣接行列 & $O(n^2)$ & $O(n^2)$  \\ \hline
隣接リスト & $O(n \omega +m )$ & $ O(n + m ) $   \\ \hline

\end{tabular}
    \begin{center}
    \small{  $n$はグラフのサイズ$|V|$,$m$はグラフの辺の数$|E|$,$\omega$はグラフの最大クリークのサイズ }
    \end{center}
\end{table}



\section{本論文の構成}
本論文の構成は,次の通りである.
%\begin{itemize}
%\item 
第2章では,グラフの基本的な定義やグラフの表現について記述する.
%\item 
第3章では,最大クリーク抽出アルゴリズムの基本的なアイデアについて説明する.
%\item 
第4章では,MCSアルゴリズムを基づく提案アルゴリズムについて説明し,その空間計算量や時間計算量について記述する.
%\item
第5章では,実験結果を記載して,結果に対して考察する.
%\item 
第6章では,本論文のまとめと今後の課題について述べる.
%\end{itemize}
